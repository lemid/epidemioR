%---- config/preamble.tex ----------------------------------------------

\usepackage[brazil]{babel}
%\usepackage[utf8]{inputenc}
%\usepackage[T1]{fontenc}

% Fontes.
\usepackage[sc]{mathpazo}
\linespread{1.05} % Palladio needs more leading (space between lines)
\usepackage{eulervm}

% \usepackage{libertine}
% \renewcommand*\familydefault{\sfdefault}

% \usepackage[sfdefault, scaled=.85]{FiraSans}
% \usepackage{newtxsf}

% \usepackage[light, math]{iwona}

% \usepackage{inconsolata}
% \urlstyle{tt}

\usepackage{booktabs}
\usepackage{epigraph}
\usepackage{wallpaper}

\usepackage{csquotes}

% https://theoryl1.wordpress.com/2016/01/15/fontawesome-in-pdftex/
% http://linorg.usp.br/CTAN/fonts/fontawesome/doc/fontawesome.pdf
\usepackage{fontawesome}

%--------------------------------------------

\usepackage{fancyhdr}

% \pagestyle{fancy}
% \pagestyle{fancy}
% \addtolength{\headsep}{25pt}
\setlength{\headheight}{18pt}
% \setlength{\footheight}{24.0pt}
\fancyhead{}
% \fancyhead[LE,RO]{\thepage}
% \fancyhead[LE]{\footnotesize{\thepage \hfill \scshape\nouppercase{\leftmark}}}
\fancyhead[LE]{\footnotesize{\thepage \hfill \scshape\nouppercase{Capítulo \thechapter}}}
% \fancyhead[RE]{\scriptsize\leftmark}
% \fancyhead[RO]{\footnotesize{\scshape\nouppercase{\rightmark}} \hfill \thepage}
\fancyhead[RO]{\footnotesize{\thesection \hfill \thepage}}
% \fancyhead[LO]{\scriptsize\rightmark}
\fancyfoot{}
% \fancyfoot[RE, LO]{www.leg.ufpr.br/wpde2013}
\fancyfoot[LO]{\footnotesize{\textbf{epidemioR}: epidemiologia de doenças de plantas aplicada com R}}
\fancyfoot[LE]{{\footnotesize \it lemid.github.io/epidemioR}}
\renewcommand{\headrulewidth}{0.4pt}
\renewcommand{\footrulewidth}{0.4pt}

\usepackage[Lenny]{fncychap}
\ChNameVar{\Large}
\ChNumVar{\Huge\itshape}
\ChTitleVar{\huge}

\pagestyle{empty}
% TODO: definir '\pagestyle{fancy}' no Rmd do primeiro capítulo.
% ATTENTION: '\pagestyle{fancy}' está sendo adicionado pela função
% `author_chapters()` definida em `setup.R`.

%--------------------------------------------

\usepackage{makeidx}
\makeindex

\renewcommand{\labelitemi}{\raisebox{0.25ex}{\footnotesize $\blacktriangleright$}}
\renewcommand{\labelitemii}{\raisebox{0.25ex}{\footnotesize $\blacktriangleright$}}
\renewcommand{\labelitemiii}{\raisebox{0.25ex}{\footnotesize $\blacktriangleright$}}
\renewcommand{\labelitemiv}{\raisebox{0.25ex}{\footnotesize $\blacktriangleright$}}

\usepackage{indentfirst}
  \setlength{\parindent}{1.2cm}
\usepackage{setspace}
  \onehalfspace
%\renewcommand{\baselinestretch}{1.5}

%\usepackage[
%  %margin=0pt,
%  %width=1\linewidth,
%  %font=small,
%  %justification=center,
%  singlelinecheck=off,
%  %format=plain,
%  %labelfont=bf,
%  %textfont=normal,
%  %labelformat=simple,
%  %format=hang,
%  up]{caption}
%\captionsetup[table]{skip=0pt, justification=raggedright, singlelinecheck=off}
%\captionsetup[table]{skip=0pt, singlelinecheck=off}
%\LTcapwidth=5.65in

\usepackage{color}
\usepackage{framed}
\setlength{\fboxsep}{.8em}
\newenvironment{note}{
  \definecolor{shadecolor}{rgb}{0, 0, 0}
  \color{white}
  \begin{shaded}
  }{
  \end{shaded}}

% \newenvironment{note}{
%   \begin{center}
%     \begin{tabular}{|p{0.6\textwidth}|}
%       \hline\\
%       }{
%       \\\\\hline
%     \end{tabular}
%   \end{center}
% }

%-----------------------------------------------------------------------
% Knitr.

% ATTENTION: this needs `\usepackage{xcolor}'.
\usepackage{xcolor}
\definecolor{color_line}{HTML}{333333}
\definecolor{color_back}{HTML}{DDDDDD}
% \definecolor{color_back}{HTML}{FF0000}

% ATTENTION: usa o fancyvrb.
% https://ctan.math.illinois.edu/macros/latex/contrib/fancyvrb/doc/fancyvrb-doc.pdf
% R input.
\usepackage{tcolorbox}
\ifcsmacro{Highlighting}{
  % Statment if it exists. ------------------
  \DefineVerbatimEnvironment{Highlighting}{Verbatim}{
    % frame=lines,     % Linha superior e inferior.
    % framesep=1ex,    % Distância da linha para o texto.
    % framerule=0.5pt, % Espessura da linha.
    % rulecolor=\color{color_line},
    % numbers=right,
    fontsize=\small, % Tamanho da fonte.
    baselinestretch=0.9,   % Espaçamento entre linhas.
    commandchars=\\\{\}}
  % Margens do ambiente `Shaded'.
  % \fvset{listparameters={\setlength{\topsep}{-1em}}}
  % \renewenvironment{Shaded}{\vspace{-1ex}}{\vspace{-2ex}}
  \renewenvironment{Shaded}{
    \vspace{2pt}
    \begin{tcolorbox}[
      boxrule=0pt,      % Espessura do contorno.
      colframe=gray!10, % Cor do contorno.
      colback=gray!10,  % Cor de fundo da caixa.
      % arc=1em,          % Raio para contornos arredondados.
      sharp corners,
      % boxsep=0.5em,     % Margem interna.
      left=3pt, right=3pt, top=3pt, bottom=3pt, % Margens internas.
      % grow to left by=0mm,
      grow to right by=6pt,
      ]
    }{
    \end{tcolorbox}
    \vspace{-3pt}
    }
  }{
  % Statment if it not exists. --------------
}

% R output e todo `verbatim'.
\makeatletter
\def\verbatim@font{\linespread{0.9}\normalfont\ttfamily\small}
\makeatother

% Cor de fundo e margens do `verbatim'.
\let\oldv\verbatim
\let\oldendv\endverbatim

\def\verbatim{%
  \par\setbox0\vbox\bgroup % Abre grupo.
  \vspace{-5px}            % Reduz margem superior.
  \oldv                    % Chama abertura do verbatim.
}
\def\endverbatim{%
  \oldendv                 % Chama encerramento do verbatim.
  % \vspace{0cm}           % Controla margem inferior.
  \egroup%\fboxsep5px      % Fecha grupo.
  \noindent{\colorbox{color_back}{\usebox0}}\par
}

%---- config/preamble.tex ----------------------------------------------
